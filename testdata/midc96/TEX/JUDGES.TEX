\documentstyle[12pt,stdpage]{article}

\begin{document}

\thispagestyle{empty}

\begin{center}
\large\bf 1996 ACM MID-CENTRAL REGIONAL\\ PROGRAMMING CONTEST\\[\bigskipamount]
\Large\bf Notes to Judges
\end{center}

\bigskip\bigskip\bigskip

Once again let me offer my thanks to those individuals who helped to
prepare this year's problem set: Jim Abele, Arkansas Tech University;
Steve Baber, Harding University; Dennis Bouvier, University of
Arkansas Fayetteville; Brad Choate, Choate Consulting; Joel DeYoung,
Seagate Software; Andrew Harrington, Loyola University Chicago; and
Frank McCown, Lockheed Martin Astronautics.

On the judge's disk is one subdirectory for each problem, and a
$\backslash$TeX subdirectory containing some support files used in
preparing the \LaTeX{} source of the problem statements.  Each problem
directory contains the \LaTeX{} source for the problem statement (a
.tex file), a DVI and Postscript version of the problem statment (.dvi
and .ps files), the problem solution submitted by the problem author
(.c, .cpp, or .pas), a compiled version of the solution, the judge's
input files (*.in files), and the judge's output files (*.out files).
In addition, some of the directories include graphics files used in
the problem statements (.cdr and .eps files), and subdirectories with
different input test cases.

Finally, some of the directories will contan a file named JUDGE.TXt
which contains information of interest to the person or persons
judging the problem.  In addition to the comments found in these
files, the following general guidelines should be observed.

\begin{enumerate}

\item Never leave a contestants floppy disk in the drive when judging
a problem.  Copy the source code to a judging directory on the hard
drive and then remove the floppy.

\item Before running a submission check the source code to be certain
that the contestants have {\em not\/} specified a drive or path in
their input and output file names.  If so, {\em reject the
submission\/} and report ``Wrong Answer'' to the team.  Teams
have been informed that this will occur.

\item Be sure to move a fresh copy of the input files for a problem to
the judging directory before judging a new submission.  Contestants
programs have been known to trash input files.

\item Ignore all output to the screen.  This allows contestants to
leave debugging code in their programs which outputs to the screen.
The contestants have been informed that screen output will be ignored
in the judging process.

\end{enumerate}

If you have questions about these or other procedures, require
clarifications or official judgments regarding contest problems, or
have errata to submit, please contact the regional chief judge on the
day of the programming contest by phone at 501-279-4562, by e-mail at
acm96@harding.edu, or via the web page at HTTP://CS.Harding.EDU/ACM96/
(with the exception of the web page, this contact information will be
valid only on the day of the contest).

Finally, let me offer a sincere thank you to all the efforts of
everyone throughout the entire region helping to make this year's
contest a great success!

Sincerely,

\bigskip\bigskip

Ronald T. Pacheco\\
Regional Chief Judge

\end{document}
