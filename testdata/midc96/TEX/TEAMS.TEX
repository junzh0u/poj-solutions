\documentstyle[11pt,stdpage]{article}

\begin{document}

\thispagestyle{empty}

\begin{center}
\large\bf 1996 ACM MID-CENTRAL REGIONAL PROGRAMMING CONTEST\\[\bigskipamount]
\Large\bf Notes to Contestants
\end{center}

\bigskip\bigskip\bigskip

\begin{enumerate}

\item Do {\em not\/} include directory or path specifications when
naming input and output files.  If a problem indicates that a files is
named ``file.in'' then the program should open ``file.in'' and not
``c:$\backslash$problem$\backslash$file.in'' or anything else.
Failure to adhere to this rule will result in a rejected run with the
reason being given as ``Wrong Answer''.

\item The judges will ignore all output to the screen.  If you write
debugging information to the screen it is not necessary to remove it
or comment it out before submitting a program.  Only output sent to
the output files named by the problem statements will be judged.

\item The output format of your submissions should adhere {\em
exactly\/} to that of the sample output shown in each problem
statement, even if the problem statement does not explicitly state
this (it is being explicitly stated here).  Note that many of the
problems will potentially be judged using file comparison utilities to
compare your output to the official answer, so it is imperative that
output specifications be followed precisely.

\item Submissions in which programs are named incorrectly or submitted
for the wrong problem will be returned as a rejected run with ``Wrong
Answer'' given as the response.

\end{enumerate}

\end{document}
