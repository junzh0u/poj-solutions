\documentstyle[11pt,../tex/prgcon]{article}

\begin{document}

\pages{4}
\pagestyle{contest}

\contest{1996 ACM MID-CENTRAL REGIONAL PROGRAMMING CONTEST}
\probno{5}
\probname{Top Dog}
\source{topdog.\{c$|$cpp$|$pas\}}
\infile{tables.in, describe.in, dispatch.in}
\outfile{dispatch.out, export.out}

\maketitle

\division{The Problem}

You are a top-secret, government-employed software engineer assigned
to TOP-DOG, the latest military intelligence program.  TOP-DOG handles
everything from mapping out enemy territory and position to parsing
highly encrypted messages from the Commander in Chief.

The operator of TOP-DOG needs to be able to transfer information from
remote computer work-stations onto diskette in case of hardware
problems or well-placed enemy fire.  All TOP-DOG information is stored
in the Oracattle database, but the only way to access the database is
through the Dispatcher, an infamous and powerful software layer that
only allows access to the database on a ``need-to-know'' basis.  You
are in charge of writing a piece of software which will get the
desired information from the Oracattle database and store it on a
``flat'' file which will later be sent out to disk.

\bigskip
\division{Input/Output Specification}

The desired tables to be exported will be stored in the file
TABLES.IN.  This file will simply contain the names of each table to
be exported on separate lines.  There will be no blank lines, and each
table name will be unique.  An example of a TABLES.IN file is shown in
the sample input section.

For each table, you must search the DESCRIBE.IN file which will
contain all column names, types, and sizes (if VARCHAR) for each table
in the Oracattle database.  This is the only information which the
Dispatcher allows you access to without begging.  An example
DESCRIBE.IN file can be found in the sample input section.

Note that, for security reasons, not every table in the TABLES.IN file
will always be available in the DESCRIBE.IN file.  Under each table
name is the column name (one unique word), data type, and size (for
VARCHAR), each separated by one space.  The \# sign indicates there
are no more columns for the table.  There will never be consecutive \#
signs immediately following each other, and each table will contain at
least one column name with the size.  The only four data types are
VARCHAR, INT, DATE, and LONGINT.

An Oracattle SQL statement must be built in order to query the
Dispatcher for the desired table data.  The Oracattle SQL statement
must be precisely built in order to keep the Dispatcher happy (we
wouldn't want the Dispatcher to be confused).  The statement begins
with ``SELECT'', followed by each column name and generic data type in
parenthesis, separated with commas, terminated with ``FROM'', the
table name, and a semicolon.  The generic data type is CHAR for
VARCHAR and DATE, and NUM for INT and LONGINT.  The generic name must
be used because the Dispatcher only understands data as CHAR or NUM
(it may be powerful, but it's not extremely intelligent).  These
SELECT statements must be put into the DISPATCH.OUT file.  An example
DISPATCH.OUT file is shown in the sample output section.

If a table name cannot be found in the DESCRIBE.IN file, a ``$<$TABLE
NOT FOUND$>$'' statement must be substituted for the SELECT statement
in the DISPATCH.OUT file.  No blank lines are to be in this file, and
only single spaces are to separate SELECT, column names, FROM, and the
table name in the SELECT statements.  The entire SELECT statement must
be on one line.

Since you currently do not have access to the Dispatcher, we will
assume that you have correctly built the DISPATCH.OUT file and that
the Dispatcher has processed it and created the table information file
DISPATCH.IN.  This file contains the table name followed by the data
from each row in the Oracattle Database table.  ``$<$NULL$>$'' is
returned for rows with empty fields.  An example DISPATCH.IN file is
shown in the sample input section.

Note that each line in DISPATCH.IN may contain any number of spaces
between words unless it was declared as a NUM.  Also, all data is
returned by row and table name in the same order it was presented to
the Dispatcher.  If no data exists in the table, a \# sign immediately
follows the table name (as in GROUPSPI in the example data).

You must now finally integrate all the information you have received
from the Dispatcher into an EXPORT.OUT file.  This file will contain
all data needed to describe the database tables.  This file will later
be imported using the Oracattle SqlImporter (OSI), a text to database
utility.  Lucky for you, all you need to do is get the EXPORT.OUT file
into the OSI format.  This can be a little tricky.  The first argument
to be supplied is the table name, followed by the number of columns in
parentheses, followed by the number of records (rows) in parentheses
(no spaces on this line).  Next comes the column name and then the
data in quotation marks (a single space should separate the column
name and it's data).  When the maximum length of the column data is
greater than 64, the size must also be supplied in parenthesis
immediately following the column name (no space in-between ).  This is
so OSI can allocate more memory for large data.  An example EXPORT.OUT
file is shown in the sample output section.

No blank lines are to exist in the EXPORT.OUT file, and all data must
remain on the same line as the column name (no end-of-line characters
in-between quotation marks).  Once this file has been created, you are
all done!

There will be a maximum of 100 columns in a single table, but there
may exist any number of rows in a single table.  The maximum column
name and table name length is 25, and the maximum data length is 100.
All ``.IN'' files will always contain data in an the expected format
(as described in these specifications), so there is no need for error
checking.  Remember, case is iMpOrTaNt- $<$NULL$>$ is not the same as
$<$null$>$.

\bigskip
\division{Sample Input}

{\bf TABLES.IN}
{\tiny
\begin{verbatim}
INTELSYS
GROUPSPI
SYSINTEL
DEPLOYREG
\end{verbatim}
}

{\bf DESCRIBE.IN}
{\tiny
\begin{verbatim}
GROUPSPI
GRCODE INT
GRSUBNET VARCHAR 20
GRREGION VARCHAR 25
GRACTION VARCHAR 100
GRREF VARCHAR 100
#
INTELSYS
ISDATE DATE
ISNUM INT
ISGEN LONGINT
ISSUBGEN VARCHAR 25
#
SYSINTEL
SITRANS INT
SISUBLET LONGINT
SINUM INT
SIGEN VARCHAR 10
SIACTION VARCHAR 50
SINOTES VARCHAR 100
#
QUICKFI
QFDATE DATE
QFDATA VARCHAR 100
#
\end{verbatim}
}

{\bf DISPATCH.IN}
{\tiny
\begin{verbatim}
INTELSYS
122922T DEC 94
1
2
<NULL>
111111Z DEC 01
3
4
CONFIRMED
010101Z DEC 02
5
6
<NOT CONFIRMED>
020202Z DEC 03
7
8
CAN'T SAY
#
GROUPSPI
#
SYSINTEL
342
3498938
000
SCOUTA
PURGE DATABASE
UNABLE TO COMPLY WITH A2DD UNDER GENERAL BURK'S COMMAND
#
\end{verbatim}
}

\newpage

\division{Sample Output}

{\bf DISPATCH.OUT}

{\tiny
\begin{verbatim}
SELECT (CHAR) ISDATE, (NUM) ISNUM, (NUM) ISGEN, (CHAR) ISSUBGEN FROM INTELSYS;
SELECT (NUM) GRCODE, (CHAR) GRSUBNET, (CHAR) GRREGION, (CHAR) GRACTION, (CHAR) GRREF FROM GROUPSPI;
SELECT (NUM) SITRANS, (NUM) SISUBLET, (NUM) SINUM, (CHAR) SIGEN, (CHAR) SIACTION, (CHAR) SINOTES FROM SYSINTEL;
<TABLE NOT FOUND>
\end{verbatim}
}

{\bf EXPORT.OUT}
{\tiny
\begin{verbatim}
INTELSYS(4)(4)
ISDATE "122922T DEC 94"
ISNUM "1"
ISGEN "2"
ISSUBGEN ""
ISDATE "111111Z DEC 01"
ISNUM "3"
ISGEN "4"
ISSUBGEN "CONFIRMED"
ISDATE "010101Z DEC 02"
ISNUM "5"
ISGEN "6"
ISSUBGEN "<NOT CONFIRMED>"
ISDATE "020202Z DEC 03"
ISNUM "7"
ISGEN "8"
ISSUBGEN "CAN'T SAY"
GROUPSPI(5)(0)
SYSINTEL(6)(1)
SITRANS "342"
SISUBLET "3498938"
SINUM "000"
SIGEN "SCOUTA"
SIACTION "PURGE DATABASE"
SINOTES(300) "UNABLE TO COMPLY WITH A2DD UNDER GENERAL BURK'S COMMAND"
\end{verbatim}
}

\end{document}

