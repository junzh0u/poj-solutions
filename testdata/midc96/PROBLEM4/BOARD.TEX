\documentstyle[11pt,../tex/prgcon]{article}

\begin{document}

\pages{3}
\pagestyle{contest}

\contest{1996 ACM MID-CENTRAL REGIONAL PROGRAMMING CONTEST}
\probno{4}
\probname{Board Silly}
\source{board.\{c$|$cpp$|$pas\}}
\infile{board.in}
\outfile{board.out}

\maketitle

\division{The Problem}

You are a member of a team of programmers whose task is to write a
board game.  Your job is to write the part that examines a board
layout and enumerates all possible moves for a given player.  The game
you are writing is played on an 8 by 8 grid of squares (similar to a
chess or checker board but the squares are all the same color).  The
rows are labeled with the letters A through H from top to bottom.  The
columns are labeled with the numbers 1 through 8 from left to right.

At any particular time in the game, each player may have from 1 to 12
pieces on the board (there is only one type of piece for each player).
Valid moves are determined by the following rules:

\begin{enumerate}

\item Pieces may be moved in a straight line, either left, right, up,
down or diagonally.

\item The number of spaces a piece may be moved is determined by the
total number of pieces in the row, column or diagonal in which the
piece is being moved.  Pieces may only be moved by this exact number,
no more, no less.

\item A player may jump over his own pieces.

\item A player may not jump over his opponent's pieces.

\item A player may capture an opponent's piece by landing on it.

\item A player may not land on a space he already occupies.

\end{enumerate}

\bigskip
\division{Input/Output Specification}

The input file consists of one or more board layouts followed by which
piece, either 'X' or 'O' (that's an ``oh'' not a zero), for which
moves should be displayed for.  Each board is specified as 8 lines of
8 characters each.  Each board position will contain an 'X', an 'O' or
a period.  The 'X' and 'O' position represent the occupied positions,
and the periods represent empty board positions.  Immediately
following each board is line containing a single 'X' or 'O' character,
and it is for this piece that moves should be displayed.  The end of
the input is indicated by the end of the file.

The output file should contain one line for each possible move.  The
moves may be printed in any order.  Each line should specify the
origin and destination of each piece, in that order, separated by a
single dash character.  Each board location (origins and destinations)
should be specified as a row letter followed by a column number.  If
no moves are possible then the output should simply specify ``No moves
are possible''.  The output for each board should be separated by a
single blank line.

\bigskip
\division{Sample Input}

{\small
\begin{verbatim}
O.......
O......X
O.....XX
O....XXX
.O...XXX
........
..O..XXX
........
O
..OXO...
..OOO...
........
........
........
........
........
........
X
\end{verbatim}
}

\bigskip
\division{Sample Output}

{\small
\begin{verbatim}
A1-A2
A1-C3
A1-E1
B1-A2
B1-B3
B1-D3
B1-F1
C1-B2
C1-C4
C1-D2
C1-G1
D1-C2
D1-D5
D1-F3
D1-H1
E2-D2
E2-D3
E2-E6
E2-G4
E2-F2
E2-F1
G3-F2
G3-F3
G3-H4
G3-H3

No moves are possible
\end{verbatim}
}

\end{document}

