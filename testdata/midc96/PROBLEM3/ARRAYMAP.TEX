\documentstyle[11pt,../tex/prgcon]{article}

\begin{document}

\pages{2}
\pagestyle{contest}

\contest{1996 ACM MID-CENTRAL REGIONAL PROGRAMMING CONTEST}
\probno{3}
\probname{Mapmaker}
\source{arraymap.\{c$|$cpp$|$pas\}}
\infile{arraymap.in}
\outfile{arraymap.out}

\maketitle

\division{The Problem}

The Cybersoft Computer Company (a leader in programming languages) has
hired you to work on a new programming language named A\verb|--|.
Your task is to work on the array mapping tasks of the language.  You
will take an array reference such as \verb'x[5,6]' and map it to an
actual physical address.  In preparation for doing this, you will
write a program that will read in several array declarations and
references and give the physical address of each reference.  The
physical address output by the program should be an integer number in
base 10.

The physical address of an array reference $A[i_1,i_2,\ldots,i_D]$ is
calculated from the formula $C_0 + C_1 i_1 + C_2 i_2 + \cdots + C_D
i_D$, where the constants $C_0\ldots C_D$ are calculated as specified
below.

\begin{tabular}{lcl}
$B$   & $=$ & Base address of the array \\
$D$   & $=$ & Number of dimensions in the array \\
$L_d$ & $=$ & Lower bound of dimension $d$ \\
$U_d$ & $=$ & Upper bound of dimension $d$ \\
$C_D$ & $=$ & Array element size in bytes \\
$C_d$ & $=$ & $C_{d+1}(U_{d+1}-L_{d+1}+1)$ for $1\le d<D$ \\
$C_0$ & $=$ & $B - C_1 L_1 - C_2 L_2 - \cdots - C_D L_D$
\end{tabular}

\bigskip
\division{Input/Output Specification}

The first line of the input file contains two positive integers.  The
first integer specifies $N$, the number of arrays defined in the data
file, and the second integer specifies $R$, the number of array
references for which addresses should be calculated.  The next $N$
lines each define an array, one per line, and the following $R$ lines
contain one array reference per line for which an address should be
calculated.

Each line which defines an array contains, in the following order, the
name of the array (which is limited to 10 characters), a positive
integer which specifies the base address of the array, a positive
integer which specifies the size in bytes of each array element, and
$D$, the number of dimensions in the array (no array will have fewer
than 1 or more than 10 dimensions).  This is followed on the same line
by $D$ pairs of integers which represent the lower and upper bounds,
respectively, of dimensions $1\ldots D$ of the array.

Each line which specifies an array reference contains the name of the
array followed by the integer indexes $i_1,i_2,\ldots,i_D$ where $D$
is the dimension of the array.

The output file should contain the array references and the physical
addresses.  There should be one array reference and physical address
per line.  The formatting guidelines below {\em must\/} be adhered to.

For each line of output:

\begin{enumerate}
\item Output the name of the array
\item Output a left square bracket
\item Output each index value (each pair of indexes should have a
single comma and space between them)
\item Output a right square bracket, a space, an equal sign, and
another space
\item Output the physical address
\end{enumerate}

\bigskip
\division{Sample Input}

{\small
\begin{verbatim}
3 4
ONE       1500 2 2 0 3 1 5
TWO       2000 4 3 1 4 0 5 5 10
THREE     3000 1 1 1 9
ONE       2 4
THREE     7
TWO       2 0 6
TWO       3 3 9
\end{verbatim}
}

\bigskip
\division{Sample Output}

{\small
\begin{verbatim}
ONE[2, 4] = 1526
THREE[7] = 3006
TWO[2, 0, 6] = 2148
TWO[3, 3, 9] = 2376
\end{verbatim}
}

\end{document}

