\documentstyle[11pt,../tex/prgcon]{article}

\begin{document}

\pages{2}
\pagestyle{contest}

\contest{1996 ACM MID-CENTRAL REGIONAL PROGRAMMING CONTEST}
\probno{1}
\probname{Polynomial Showdown}
\source{poly.\{c$|$cpp$|$pas\}}
\infile{poly.in}
\outfile{poly.out}

\maketitle

\division{The Problem}

Given the coefficients of a polynomial from degree 8 down to 0, you
are to format the polynomial in a readable format with unnecessary
characters removed.  For instance, given the coefficients 0, 0, 0, 1,
22, -333, 0, 1, and -1, you should generate an output line which
displays \verb|x^5 + 22x^4 - 333x^3 + x - 1|.

The formatting rules which must be adhered to are as follows:

\begin{enumerate}

\item Terms must appear in decreasing order of degree.

\item Exponents should appear after a caret ``\verb|^|''.

\item The constant term appears as only the constant.

\item Only terms with nonzero coefficients should appear, unless all terms
have zero coefficients in which case the constant term should appear.

\item The only spaces should be a single space on either side of the
binary $+$ and $-$ operators.
    
\item If the leading term is positive then no sign should precede it;
a negative leading term should be preceded by a minus sign, as in
\verb|-7x^2 + 30x + 66|.

\item Negated terms should appear as a subtracted unnegated term (with
the exception of a negative leading term which should appear as
described above).  That is, rather than \verb|x^2 + -3x|, the output
should be \verb|x^2 - 3x|.

\item The constants 1 and -1 should appear only as the constant
term.  That is, rather than \verb|-1x^3 + 1x^2 + 3x^1 - 1|, the output
should appear as \verb|-x^3 + x^2 + 3x - 1|.

\end{enumerate}

\bigskip
\division{Input/Output Specification}

The input file will contain one or more lines of coefficients
delimited by one or more spaces.  There are nine coefficients per
line, each coefficient being an integer with a magnitude of less than
1000.  The output file should contain the formatted polynomials, one
per line.

\newpage

\division{Sample Input}

{\small
\begin{verbatim}
0    0    0    1   22 -333    0    1   -1
0    0    0    0    0    0  -55    5    0
\end{verbatim}
}

\bigskip
\division{Sample Output}

{\small
\begin{verbatim}
x^5 + 22x^4 - 333x^3 + x - 1
-55x^2 + 5x
\end{verbatim}
}

\end{document}
